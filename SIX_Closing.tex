%! suppress = TooLargeSection
%! suppress = LineBreak
%! Author = LoMueller
%! Date = 03.08.2020

\documentclass[11pt,a4paper, notitlepage]{article}

% GENERAL PACKAGES
\usepackage[top=3cm, bottom=3cm, left=3cm, right=3cm]{geometry}
%\usepackage{geometry}
\linespread{1.5}
\usepackage{setspace}
\usepackage{enumerate}
\usepackage{amssymb}
\usepackage{amsmath}
\usepackage{amsfonts}
\usepackage[utf8]{inputenc}
\usepackage[english]{babel}
\usepackage[all]{nowidow}
\usepackage[bottom]{footmisc} \interfootnotelinepenalty=10000
\usepackage{xurl}
\usepackage{comment}
%\usepackage[acronym]{glossaries}

% REFERENCES
\usepackage{float} % load before hyperref
\usepackage[printonlyused, nohyperlinks, withpage]{acronym}
\usepackage[backend=biber, style=apa]{biblatex}
\usepackage{csquotes}
\addbibresource{Dissertation.bib}
\usepackage[colorlinks, citecolor=red, linkcolor=red, urlcolor=blue, linktocpage]{hyperref} % ocgcolorlinks
\usepackage[noabbrev]{cleveref}

% FIGURES / TABLES
\usepackage{tabularx}
\usepackage{booktabs}
\usepackage{graphicx}
%\graphicspath{{Report_Figures/}} % Path to all images
%\usepackage[labelfont=bf,labelsep=colon,font={footnotesize,singlespacing, sf}]{caption}
\usepackage[labelfont=bf,labelsep=colon,font={footnotesize,singlespacing}]{caption}
\usepackage[labelformat=simple,labelsep=colon,font={large}]{subcaption}
\usepackage{xcolor}


%\newcolumntype{L}[1]{>{\raggedright\let\newline\\arraybackslash\hspace{0pt}}m{#1}}
%\newcolumntype{C}[1]{>{\centering\let\newline\\arraybackslash\hspace{0pt}}m{#1}}
%\newcolumntype{R}[1]{>{\raggedleft\let\newline\\arraybackslash\hspace{0pt}}m{#1}}
%\newcolumntype{R}{>{\raggedleft\arraybackslash}X}
%\newcolumntype{Y}{>{\centering\arraybackslash}X}

%\usepackage{pdflscape}
%\usepackage{array}


%\newacro{SIX}{SIX Securities \& Exchanges}
\newacro{SIX}{SIX Swiss Exchange}
\newacro{ior/cf}{Institute for Operations Research anc Computational Finance}
\newacro{WPDC}{Weighted Price Discovery Contribution}
\newacro{ECN}{electronic communication network}
\newacro{MTF}{multilateral trading facilitie}
\newacro{HFT}{high-frequency trading}
\newacro{SI}{systematic internaliser}
\newacro{SLI}{Swiss Leadership Index}

\title{Liquidity-related Price Sensitivities of Closing Auctions in Equity Markets\thanks{This work has been motivated by an ongoing research cooperation between \acf{SIX} and the Institute for Operations Research and Computational Finance (ior/cf-HSG) at the University of St.Gallen. We conducted research on several aspects of market microstructure. Most prominently discussed were topics of closing auctions and exchange performance in context of the European trading landscape. We would hereby like to thank SIX for their valuable inputs and for supplying detailed and granular order-level data of the Swiss stock exchange.}}

\author{Karl Frauendorfer
%\thanks{University of St. Gallen, Institute for Operations Research and Computational Finance (ior/cf-HSG), Switzerland.}
\and Louis Müller
%\footnotemark[2]
}

\date{\today}


\begin{document}
	
	\begin{titlepage}
		
		\centering
		
		
		\maketitle

%		\vspace{1cm}
		\emph{Working Paper} \\
		
		\vspace{1.5cm}
		\textsc{ \normalsize Institute for Operations Research and Computational Finance \\ School of Finance \\ University of St. Gallen }
		
		
		\vfill
		
		\begin{abstract}
			In current equity markets where closing auctions capture an increasing share of overall volume, price accuracy is paramount. Competition for order flow may be harmful for the price discovery process throughout the auction. In this paper, we analyse closing auctions of Swiss equities and show that closing prices are very sensitive towards removals of small percentages of liquidity. This is true for both limit- and market orders individually. We also show that closing auctions contribute significantly to open-to-close price discovery, i.e.\ 13--24\% on the three most heavily traded auction days. By looking at auctions at a more granular level, we show that almost all of the price discovery happens within the first two minutes of the auction. The three largest stocks of the sample contribute around 16\% of the overall price discovery during the closing auction within the first 30 seconds.
		\end{abstract}

%		\vspace{0.5cm}

%		\renewcommand{\abstractname}{Acknowledgement}
%		\begin{abstract}
%			
%		\end{abstract}
		
		\thispagestyle{empty}
	\end{titlepage}

%
	\clearpage
	\pagenumbering{arabic}
	
	
	\section{Introduction} \label{sec:introduction}
	
	Throughout exchanges in Europe, increases in trading volumes at the close have been observed consistently. On Euronext Paris for instance, CAC40 stocks have been trading more than 40\% of their volume during closing auctions in the year 2019~\parencite{Raillon2020}. This observation can be explained by four main developments:
	
	\begin{enumerate}[(1)]
		\item \textbf{Passive investing} (e.g.\ through ETFs) has seen large inflows in recent years. Closing prices provide a precise benchmark on how these funds must re-balance their positions. Moreover, index funds engaging in huge block trades require the liquidity during the closing auction.
		\item \textbf{Best-execution requirements}, that were introduced under the Markets in Financial Instruments Directive (MiFID) II (effective early 2018), force brokers to trade in the best interest of their clients, i.e.\ with the lowest trade-costs. Since closing prices are determined in one market exclusively, brokers do not have to compare quotes across different venues. Instead, they can be more certain to get the best price on behalf of their clients.
		\item \textbf{Adverse selection} is an important consideration for traders during the day. In this context, it refers to the situation where some market participants have private information and are therefore making a profit at the expense of less informed traders. This is particularly problematic during the continuous trading phase with high market fragmentation. In this case, it is possible that the same security has different prices on other platforms for very short periods of time~\parencite{BudishCramtonShim2015}. Predatory \ac{HFT} strategies based on speed can thrive in those conditions~\parencite{BiaisFoucaultMoinas2015}.
		\item \textbf{Execution algorithms} are learning that there are better execution opportunities during closing, since there is less adverse-selection (according to (3)). This in turn emphasizes end-of-day trading even stronger, eventually triggering a positive feedback-loop with liquidity clustering in those auctions~\parencite{Pagano1989Volume}.
	\end{enumerate}
	
	Such increasing importance of closing auctions with respect to intraday trading also raises certain questions. First and foremost, when more volume shifts into these auctions price discovery becomes more important, particularly because the closing price marks the \emph{reference price} to many market participants~\parencite{KandelRindiBosetti2012}. At the same time, it becomes more attractive for other venues to capture a chunk of that volume. There are two types of venues that provide alternative trading mechanisms to trade at the closing price. First, there are \acp{SI} who trade their own inventory and may take speculative positions on the main exchange after having insight into their order flow prior to the auction. These \acp{SI} are often an integrated part of investment banks and allow for less costly execution compared to the main exchange. Second, there are \acp{MTF} acting as competing trading platforms, who seek to take market share from the main exchange.  Despite such behaviour representing healthy competition between trading venues, there may be adverse consequences associated with it. For instance, the universal \emph{reference price} for all other market closings is determined solely on the respective primary exchange. However, accurate price discovery crucially depends on the accumulation of all the participating volume, which in turn reveals the aggregation of all information to the entire market~\parencite{Madhavan1992}.
	
	The question about market fragmentation and price discovery has been a very important one in academic literature in recent years. Despite several contradicting findings in some papers, the overall consensus views market fragmentation as positive for market quality if investors have simultaneous access to all venues and are interested in trading liquid large-cap stocks. Some relevant papers on this topic include \textcite{GomberSagadeTheissenWeberWestheide2017}, \textcite{HaslagRinggenberg2016}, \textcite{PaganoPengSchwartz2013}, \textcite{AitkenChenFoley2017}, \textcite{DegryseDeJongVanKervel2015}, \textcite{OharaYe2011}. Despite the overall favourable view with respect to fragmentation, all of these papers look at continuous trading in isolation. which does not directly apply to closing auctions. As \textcite{Madhavan1992} pointed out, call auctions have entirely different trading mechanisms and therefore attract a different type of investor to participate.
	
	To shed light on the question of market quality at closing, one needs to first understand and study the composition of closing order books. This paper analyses equity markets through granular order and trade data provided by \acf{SIX}. The analysis is based on the entire year of 2019 with 249 trading days. Included are all 30 constituents of the \ac{SLI} on the last day of the sample period. Therefore, our sample of equities only contains highly liquid Blue Chips. In order to facilitate any of our analyses, we first had to reconstruct the closing order book recursively based on all incoming orders throughout the day. The same was done for throughout the closing auction, which last 10 minutes on \acf{SIX}. More specifically, we reconstructed snapshots of the full order books at the following timestamps: (1) At the end of continuous trading, (2) at the beginning of closing auction, (3) in 30 second intervals throughout the auction (resulting in 20 intervals) and (4) at the very end of the auction (just before the close). Based on these reconstructions of closing order books, we have derived three core pieces of analysis:
	\begin{enumerate}[(i)]
		\item Sensitivity analysis of order books.
		\item Price dislocations and discovery at close.
		\item Price discovery over the course of the auction.
	\end{enumerate}
	
	This note presents our descriptive results and is structured as follows: In \cref{sec:sensitivities}, we will show how sensitive certain stocks behave in terms of the removal of liquidity from the top of the book. Subsequently, \cref{sec:Discovery} will broadly scrutinize the dislocations of closing prices from the end-of-day trading prices and scrutinizes the price discovery process over the course of the closing auction. Finally, \cref{sec:outlook} presents the conclusion and an outlook of potential further research in the realm of closing auctions.
	
	
	\section{Order book sensitivities} \label{sec:sensitivities}
	
	\subsection{Methodology}\label{subsec:methodology}
	
	In a first step, we show how sensitive order books behave with respect to an outflow of liquidity. In an increasingly fragmented closing market, some of the liquidity will flow out of the main exchange into other trading venues. This potentially has an impact on the closing price, making it more unstable and less capable to serve large one-sided pressure.
	
	For this reason, the sensitivity analysis is simulating the outflow of orders underlying a certain logic. The resulting price deviations from the original closing are subsequently analyzed more in-depth. All of our methods to remove liquidity have certain rules in common:
	\begin{enumerate}
		\item Liquidity is removed from the top of the order book first. This prioritizes market orders over limit orders, as they are willing to buy or sell at any price.
		\item All of the price deviations are reported in absolute terms, to enhance comparability. This is mostly an adjustment to the removal of bid liquidity, as this will lower the price, based on simple demand and supply economics.
		\item  The stocks are split into three terciles based on trading volume. These quantiles are re-assigned on a daily basis, hence allowing for stocks to change quantiles over time after becoming comparatively more or less frequently traded.
	\end{enumerate}
	
	After establishing these rules, we simulate the impact of removal a certain percentage (i.e. 5--35\%) of liquidity from the order book. We present three different ways to compute the percentage base:
	
	\paragraph{Separate liquidity basis}
	The removal of liquidity is based on the combination of market- and limit orders \emph{individually} for both sides of the book. This represents the entire volume in the market of trading participants either willing to buy or sell, regardless of the price. Under this algorithm, the total volumes (i.e. the percentage bases) are calculated separately for each side of the book. Consequently, the outflow of a given percentage of liquidity removes unequal amounts in currency terms, taking into account arising order imbalance. Moreover, it is possible that after the removal of liquidity there are no crossing limit orders left to find an uncrossing price.
	
	\paragraph{Symmetric liquidity basis}
	The removal of liquidity is based on the \emph{average} of market and limit orders for both sides of the book. In contrast to the previous algorithm, both sides of the book are assigned the same percentage base. Therefore, we ignore order imbalances under this algorithm, by always removing the same CHF amount from both sides of the book. As before, it is possible that this algorithm removes all limit orders necessary to form a crossing price.
	
	\paragraph{Execution volume basis}
	In contrast to the other two bases, this one focuses only on the volume that has been executed. By definition, this volume must be equal on both sides of the markets, regardless of potential order imbalances. Importantly, the execution volume is only a small subset of all the market- and limit orders at the closing and highly dependent on the structure of the order book and on the outcome of the auction.  \\
	
	Combined, those three ways to compute percentage bases give a diverse insight into various scenarios of liquidity outflow. It is important to keep in mind, that even though the liquidity removed from both sides of the book might be identical, the effects on the closing price may not be. The reason for this lies in the \emph{slope} of the order books, which implies how far from the optimal uncrossing price most of the liquidity is located. All of these scenarios are presented in the following subsections.
	
	\subsection{Results of liquidity removal} \label{subsec:results-of-liquidity-removal}
	
	To begin with, \cref{fig:SensLimSeparate} shows all the distributional results for the \emph{separate liquidity} case. In this case, liquidity percentages are removed individually from both sides of the book. It can be seen that the ask side is more sensitive towards liquidity removal, particularly for stocks with comparatively lower transaction volume. This phenomenon has two explanations. First, based on the way liquidity removal is calculated, it is possible that there is more liquidity is on the ask- than the bid side. Second, liquidity tends to be further away from the optimal uncrossing price on the bid side, making it more sensitive towards outflow from top-of-book liquidity. The symmetric removal of liquidity from both sides reveals that bid orders are slightly more sensitive than ask orders.
	
	
	\begin{figure}[tp]
		\centering
		\includegraphics[trim= 3mm 3mm 3mm 3mm, clip=True,width=1\textwidth]{06 Figures/Sens_limit_SeparateOrders.pdf}
		\caption{Distribution of absolute price impact caused by liquidity removal via \textit{separate liquidity algorithm}. The horizontal axis represents the percentage of liquidity removed and the vertical axis represents the absolute price deviation in basis points. Panel A(B) shows the impact of percentage liquidity removal of bid(ask) limit orders. Both panels share the same scaling of the vertical axis. For Panel C, liquidity was removed symmetrically. The three liquidity terciles are re-assigned daily and are based on trading volume.}
		\label{fig:SensLimSeparate}
	\end{figure}
	
	\begin{figure}[tp]
		\centering
		\includegraphics[trim= 3mm 3mm 3mm 3mm, clip=True,width=1\textwidth]{06 Figures/Sens_limit_FullLiquidity.pdf}
		\caption{Distribution of absolute price impact caused by liquidity removal via \textit{symmetric liquidity algorithm}. The horizontal axis represents the percentage of liquidity removed and the vertical axis represents the absolute price deviation in basis points. Panel A(B) shows the impact of percentage liquidity removal of bid(ask) orders. Both panels share the same scaling of the vertical axis. The three liquidity terciles are re-assigned daily and are based on trading volume.}
		\label{fig:SensLimSymmetric}
	\end{figure}
	
	\begin{figure}[tp]
		\centering
		\includegraphics[trim= 3mm 3mm 3mm 3mm, clip=True,width=1\textwidth]{06 Figures/Sens_limit_CrossedVolume.pdf}
		\caption{Distribution of absolute price impact caused by liquidity removal via \textit{execution volume algorithm}. The horizontal axis represents the percentage of liquidity removed and the vertical axis represents the absolute price deviation in basis points. Panel A(B) shows the impact of percentage liquidity removal of bid(ask) orders. Both panels share the same scaling of the vertical axis. The three liquidity terciles are re-assigned daily and are based on trading volume.}	
		\label{fig:SensLimVolume}
	\end{figure}
	
	In contrast to this, \cref{fig:SensLimSymmetric} visualizes the effects of a removal based on \emph{symmetric liquidity}, where the same CHF amount is deducted from both sides of the book. This has the advantage that it only considers the so-called \emph{slope} of the book. Considering the liquidity terciles, there does not seem to be much of a difference in terms of distribution, with the exception of a few illiquid stocks that are much more sensitive than the rest. This is why under a 35\% removal of liquidity, the 95\% quantile reaches up to a deviation of 500 bps from the original closing price.
	
	The final algorithm including limit orders is shown in \cref{fig:SensLimVolume} based on \emph{execution volume}. Naturally, the base for the liquidity removal is much smaller than for the other two methods, due to not considering un-executed volumes. For this reason, the scale is smaller in comparison to \cref{fig:SensLimSeparate,fig:SensLimSymmetric}. Interestingly, \cref{fig:SensLimVolume} shows how large cap stocks are the most sensitive with respect to an outflow of liquidity. This comes from the fact that very much liquid titles tend to have more market orders submitted, which are executed at any price. With less liquid stocks, investors seem to be more cautious and therefore prefer using limit orders due to fear of extreme adverse price movements.
	
	\begin{figure}[tp]
		\centering
		\includegraphics[trim= 3mm 3mm 3mm 3mm, clip=True,width=1\textwidth]{06 Figures/Sens_mkt.pdf}
		\caption{Distribution of absolute price impact caused by removal of market orders. All of the depicted scenarios remove the entirety of relevant market orders in the book. Panel A shows the sensitivities when all market orders are removed. Panel B only removes market orders that have been submitted before the beginning of the closing auction.}
		\label{fig:SensMkt}
	\end{figure}
	
	\Cref{fig:SensMkt} is structured slightly different to the prior results. In that we have simulated the full removal of all market orders. Therefore, the figure only shows how the auction would have uncrossed without considering market orders. The results are somewhat similar to \cref{fig:SensLimVolume}, indicating that for the most liquid stocks, market orders at the close have higher importance than for less liquid ones. Consequently, the removal of all market orders makes liquid stocks more sensitive in terms of price deviations. By removing all market orders, from each side of the market individually, we receive a median price deviation of around 50 bps. When only considering the market orders that were submitted before the auction during the continuous trading, the effects are much smaller and mostly below 5 bps.
	
	
	\section{Price Discovery} \label{sec:Discovery}
	
	In order to determine whether fragmentation of closing auctions has detrimental effects on price discovery, we quantified the process of price discovery throughout the closing auction. To achieve this, we first need to define several measures that are relevant for the later presentation of results.
	
	In the first part of this section, a measure of order imbalance is required to assess whether anyone having private information about the incoming order flow would receive a signal to subsequently trade on. In academic literature, there has been some evidence that order imbalance can lead price movements. Examples of this include \textcite{ChordiaRollSubrahmanyam2008,ChordiaRollSubrahmanyam2005}. It is yet unclear, however, how order imbalances at various points during the closing auction influence the outcome. For our purposes, order imbalances ($IMBAL$) at the start of the auction are computed following \textcite{HoldenJacobsen2014} for each stock $s$ on day $d$, where:
	\[ IMBAL_{d,s} = \frac{VOL^{bid}_{d,s} - VOL^{ask}_{d,s}}{VOL^{bid}_{d,s} + VOL^{ask}_{d,s}}. \]
	Based on this formula, the result is bounded by $IMBAL \in (-1,1)$. The volumes $VOL$ are computed based on the number of order in the closing order book times the closing price \emph{at the beginning of the auction}. This way, we visualize certain patterns on the predictability of closing returns prior to the auction.
	
	\subsection{Contribution to the Trading Day} \label{subsec:OverallDiscovery}
	
	This part is concerned with price discovery, which is approximated using the \ac{WPDC} measure, used by \textcite{BarclayWarner1993}, \textcite{BarclayHendershott2003}. Generally speaking, this measure considers two trading periods, one longer and one shorter, where the latter a strict sub-period of the former. In our application in this paper, we first compute how much the closing auction contributes to the overall trading day in terms of price discovery. For this purpose, we define $ret_{d,s}$ as the logarithmic open-to-close return on day $d$ in stock $s$. Similarly, the logarithmic closing return is defined as $ret^{CL}_{d,s}$. The ratio between both we define as
	\begin{equation*}
		PDC^{CL}_{d,s} = \frac{ret^{CL}_{d,s}}{ret_{d,s}}
	\end{equation*}
	which represents the price discovery contribution of the closing return with respect to the open-to-close return. For stability reasons, days with open-to-close returns within the interval $(-\varepsilon, \varepsilon)$ are disregarded, where $\varepsilon$ is a sufficiently small number. Due to stocks experiencing correlated shocks on the same trading days, this measure is extended by a weighting term. This weighting term is defined per stock and day as the fraction between the absolute open-to-close return $|ret_{d,s}|$ and the sum of absolute returns across all 30 stocks on the same day:
	\begin{equation}
		\label{eq:wpdc_base}
		WPDC^{CL}_{d,\hat{s}} = \underbrace{\frac{|ret_{d,\hat{s}}|}{\sum_{s=1}^{S} |ret_{d,s}|}}_{weighting}  \times \underbrace{PDC^{CL}_{d,\hat{s}}}_{contribution}.
	\end{equation}
	Following this definition, the weights must sum up to 1, while giving more weight to stocks with large open-to-close returns compared to the remaining stocks. Therefore, stocks with large open-to-close returns will obtain a higher weighting than small ones. The second component constitutes the contribution term, which in our case divides the closing return by the open-to-close return. The resulting value is positive with the closing return going into the same direction as the open-to-close return and negative otherwise. In order to aggregate the computed $WPDC$ values by day or stock, two more measures are derived in context of this analysis. First, \cref{eq:wpdc_day} takes advantage of the fact that the weighting for each day sum up to one.
	\begin{equation}
		\label{eq:wpdc_day}
		WPDC^{CL}_{d} = \sum_{\hat{s}=1}^{S} \left( \frac{|ret_{d,\hat{s}}|}{\sum_{s=1}^{S} |ret_{d,s}|} \times \, PDC^{CL}_{d,\hat{s}} \right)
	\end{equation}
	This entails that the daily sum of $WPDC$ represents a weighted average of all price discovery contributions of individual stocks within each trading day. Second, we aggregate the price discovery contributions $PDC^{CL}_{d,s}$ on a stock basis. For this purpose, \cref{eq:wpdc_stock} effectively calculates the t-statistic given the null hypothesis of no price contribution on average $H_0: \mathbb{E}\left[  PDC^{CL}_{s} \right] = 0$. To achieve this, we define $\sigma (X) = \sqrt{ \text{Var} \left[ X \right] }$ and $D$ as the total number of days.
	\begin{equation}
		\label{eq:wpdc_stock}
		TPDC^{CL}_{s} =  \frac{\mathbb{E} \left[  PDC^{CL}_{s} \right] }{ \sigma \left( PDC^{CL}_{s} \right) \; / \; \sqrt{D}}
	\end{equation}
	This is achieved by taking the expectation of $PDC$ for each stock $s$ across a total of $D$ days and subsequently dividing by the standard error.
	Since we are essentially calculating the t-statistic, this measure is named $TPDC$.
	
	
	The first part of this analysis aims to show some of the implications of order imbalance with respect to price discovery measures. \Cref{fig:Discovery} presents three scatter plots showing the relationship between different aggregations of the $WPDC$ measure and the order imbalance $IMBAL$. It attempts to visualize the distribution of price discovery and reveal potential patterns inside the order flow.
	
	\begin{figure}[htp]
		\centering
		\includegraphics[width=1\textwidth]{06 Figures/Discovery_Scatter.pdf}
		\caption{\ac{WPDC} and order imbalance. Panel A plots all stock-day observations in the sample as defined in \cref{eq:wpdc_base}. Panel B shows $WPDC^{CL}_d$ aggregated on a daily basis as defined in \cref{eq:wpdc_day}. The horizontal and vertical axes have been adjusted for each panel. Panel C represents the t-statistic of the price discovery measure $TPDC^{CL}_s$ as defined in \cref{eq:wpdc_stock}. Color and size of the circles are determined by the average trading volume throughout the sample. The shaded area represents the area of non-significance at the 5\% level for $| TPDC | < 1.96$. }
		\label{fig:Discovery}
	\end{figure}
	
	Panel A represents all stock day observations of the dataset. At first sight there does not seem to be a strong pattern between both of these measures. However, most of the observations with turnover of more than CHF 400 million in the closing alone have a positive $WPDC^{CL}_{d,s}$ as defined in \cref{eq:wpdc_base}. This indicates that observations with large trading volume tend to move into the same direction as the overall return during the day and not in the opposite direction. This implies that the closing auctions for these stock-days have a positive contribution to price discovery.
	
	Panel B shows the aggregated $WPDC^{CL}_d$ on a daily basis by accumulating the weighted average as shown in \cref{eq:wpdc_day}. In this panel, we can clearly see that the days with the largest closing turnovers (i.e. the quarterly future expiration dates) have clearly positive momentum within the trading day. On these days, the closing auction contributed 18.4\% (March), 24.4\% (June), 13.2\% (October) and 7.2\% (December) to the open-to-close price discovery across stocks. Most of the observations with negative WPDC are related to low aggregated turnover.
	
	Finally, Panel C of \cref{fig:Discovery} visualizes the t-statistics of contributions by individual stocks $TPDC^{CL}_s$. Importantly, each of the circles represents one SLI stock, whereas the size and shading of the circle are ordered by average trading volume throughout the sample period. The largest dots in this panel seem to be distributed at random, without revealing a clear pattern. There are only two stocks with significantly positive price discovery contribution, that are SOON and ALC with values of 2.478 and 2.403. On the contrary side, SREN has the lowest t-statistic with $-1.733$, which is still significant at the 10\% level.
	
	\subsection{Incremental Price Discovery} \label{subsec:intervals}
	
	The analysis about the WPDC has been extended by a more granular approach to closing auctions. So far, \cref{subsec:OverallDiscovery} only looked at closing auctions as a whole with respect to the trading day. In this section, we are splitting the closing auction into 20 equally-spaced 30-second intervals to cover the entirety of the 10-minute auction.
	
	\Cref{fig:InterDevs} represents the median price deviations for the 8 largest stocks in our sample (Panel A) as well as for each individual month in the sample (Panel B). Price deviations are defined as the difference between the hypothetical closing price at the end of each interval and the realized closing price at the end of the auction. The fact that the \emph{median} is deviating quite significantly (both positively and negatively) throughout the auction shows that there is substantial price discovery going on throughout the auction. This is particularly visible in Panel B, which contains all 30 stocks of the sample. The largest 8 stocks are more stable throughout the closing auction with maximum median deviations of 20bps. Naturally, all of the curves are expected to converge to zero over the course of the auction.
	
	\begin{figure}[!t]
		\centering
		\includegraphics[trim= 3mm 3mm 3mm 3mm, clip=True,width=1\textwidth]{06 Figures/Interval_Deviations.pdf}
		\caption{Price deviation throughout the closing auction. Price deviation is defined as the difference between the hypothetical closing price at the end of each 30-second interval and the final realized closing price. Panel A represents median price deviations for the 8 most liquid SLI stocks. Panel B shows the median price deviation for each month across all 30 SLI stocks in 2019.}
		\label{fig:InterDevs}
	\end{figure}
	
	This section aims to extend the results of the previous section. For this purpose, we slightly adjust the price discovery measures as derived in \cref{eq:wpdc_base}. In this analysis, we look at the contribution of each interval $i$ individually. Therefore, we additionally define $ret_{d,s,i}$ as the logarithmic interval returns and $I$ as the total number of intervals within closing auctions, such that
	\begin{equation*}
		ret^{CL}_{d,s} = \sum_{i=1}^{I} ret_{d,s,i}
	\end{equation*}
	due to the logarithmic nature of returns. Following this, the interval-based price discovery contribution is now defined as the ratio between each interval return and the closing return. Consequently, we slightly adjust the definitions in \cref{eq:wpdc_base,eq:wpdc_day} to
	\begin{gather}
		\label{eq:ival_wpdc_base}
		WPDC^{IV}_{d,\hat{s},i} = \frac{|ret^{CL}_{d,\hat{s}}|}{\sum_{s=1}^{S} |ret^{CL}_{d,s}|}  \times \frac{ret_{d,\hat{s},i}}{ret^{CL}_{d,\hat{s}}} \\[5mm] \label{eq:ival_wpdc_day}
		WPDC^{IV}_{d,i} = \sum_{\hat{s}=1}^{S} WPDC^{IV}_{d,\hat{s},i}
	\end{gather}
	
	
	\begin{figure}[h!t]
		\centering
		\includegraphics[trim= 3mm 3mm 3mm 3mm, clip=True,width=1\textwidth]{06 Figures/Interval_WPDC.pdf}
		\caption{\ac{WPDC} throughout the closing auction. In this figure, $WPDC$ is defined as in \cref{eq:ival_wpdc_base,eq:ival_wpdc_day}. Panel A represents the average $WPDC$ for the 8 most traded stocks in the SLI. Panel B shows the median $WPDC$ by month and interval across all 30 SLI stocks in 2019.}
		\label{fig:InterWPDC}
	\end{figure}
	
	
	\Cref{fig:InterWPDC} shows the evolution of the interval-based $WPDC^{IV}_{d,\hat{s},i}$ over the course of the auction. Panel A represents the average price discovery process for a selection of large cap stocks. For this purpose the average was chosen, as there were not many outliers and the median was essentially zero, since most of the intervals do not realize any change in the hypothetical closing price. Nonetheless, the figure shows clearly that most of the price discovery is happening within the first two minutes after the beginning of the auction, particularly the first. After the two-minute mark, the activity quickly dies down. This indicates that the closing return for the most heavily traded stocks is mostly realized very early in the auction. Particularly the three largest stocks NESN, NOVN and ROG contribute on average 5--6\% to the price discovery across all 30 stocks in the first 30 seconds of the auction.
	
	A similar conclusion can be drawn from Panel B, where we present the median $WPDC^{IV}_{d,i}$ as defined in \cref{eq:ival_wpdc_day} aggregated by month. The reason for choosing the median lies in the fact that many small cap stocks have large swings throughout the auction caused by a lack of liquidity. These extreme observations distort the mean for the remaining observations. Nonetheless, most of the price discovery happens in the first two minutes of the auction, irrespective of the trading month. Particularly interesting is the month of July, where the median price discovery in the 30--60 second interval is almost 100\%. Overall, the graphs in this panel suggest that stocks that move during the auction do so in the first two minutes. Afterwards, the median WPDC remains between positive and negative 30\%.
	
	
	\section{Conclusions and Outlook} \label{sec:outlook}
	
	With the increasing relevance of closing auctions, it is clear that the price discovery process must be accounted for when considering additional fragmentation at the close. We have shown that closing prices can be quite sensitive with respect to outflowing liquidity. Particularly, the simulation of one-sided removal of only 25\% of the execution volume from each side of the market entails a median dislocation of 20bps for all liquidity terciles. Similarly, a removal of all market orders displaces the closing price by around 50 basis points for both bid- and ask market orders. Interestingly, the most liquid stocks seem to be the most sensitive towards an outflow of liquidity based on execution volume. In a next step, we analyzed the price discovery on both the level of the full auction as well as over the course of the auction. For approximation purposes, we implemented several distinguished measures of \acf{WPDC} in conjunction with order imbalances prior to the auction. On days with large closing volume, closing auctions tend to reinforce intraday returns and are therefore beneficial to price discovery. In particular, the three most traded closing auctions (which happen to be the future expiration dates in March, June and September) contribute between 13 and 24\% to open-to-close price discovery. Finally, we showed how the price discovery process is varying throughout the auction by extracting 30-second intervals. The evidence shows that the majority of price discovery takes place within the first two minutes of the auction. After this point, prices are still adjusting, but in a noisy way without benefit to price discovery. Moreover, the three largest constituents of the 30 \ac{SLI} stocks contribute on average 5--6\% each to the total price discovery during the auction across all stocks.
	
	These observations indicate two features of closing auctions. First, the auction price is sensitive towards even small removals of liquidity. Second, most price discovery happens in the first two minutes of the auction. Both of these observations imply certain risks of fragmentation of closing auction liquidity across venues. For instance, if a third-party platform diverts liquidity from the closing auction on the main exchange, prices may become increasingly volatile, as the closing price is determined over a smaller set of orders. Similarly, by diverting the order flow from the beginning of the auction, an investment bank may take positions towards the end of the auction due to exclusive visibility on its order flow. All of these opportunities would jeopardize overall market quality.
	
	Despite our findings showing certain risks of closing auction fragmentation, there is much potential for further research into this topic. We mainly see two areas that may be further explored in the future. The first is about the predictive power of closing order flow before the start of the auction. It has well been documented in academic literature, that order flow carries information about future price movements (e.g. in~\textcite{ChordiaRollSubrahmanyam2005,ChordiaRollSubrahmanyam2008}). This predictive power is usually only over very short time horizons in the intraday market. However, this phenomenon has not yet been adequately researched for closing auctions. An approach like this would quantify the degree to which private information about the order flow may be profitable in economic terms.
	
	The second area deals with the question of what drives participation in closing auction primarily. It can be argued that some investors make a deliberate choice about when to enter or exit their positions, be it human- or machine-driven. Many investors nowadays make use of trading algorithms that autonomously decide when market conditions are most favourable for execution. There has been the notion that trading algorithms cluster together to trade with each other to match their executions during maximum liquidity. If it is true that market participants generally opt for the closing auction on tumultuous trading days to avoid any type of execution risk, the accuracy of closing prices becomes increasingly important for them. There is an argument to be made that if investors view those auctions as some sort of \emph{safe haven} for execution, jeopardizing it through fragmentation would be harmful overall. Conclusively, it is important to rigorously understand what drives investors' decisions about whether to enter closing auctions or trade before in order to make an argument for maximum visibility and consolidation in those auctions.
	
%	\clearpage
	
	\printbibliography[heading=bibintoc]

\end{document}